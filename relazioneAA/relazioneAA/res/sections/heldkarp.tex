\section{Risultati degli algoritmi}
Questa sezione risponderà alla Domanda 1: verranno riportati sotto forma di tabella i risultati dei costi per il problema TSP dei grafi richiesti, il tempo di esecuzione di ogni algoritmo e l'errore relativo rispetto alla soluzione esatta.

\subsection{Domanda 1}
Eseguite i tre algoritmi che avete implementato (Held-Karp, euristica costruttiva e 2-approssimato) sui 13 grafi del dataset. Mostrate i risultati che avete ottenuto in una tabella come quella sottostante. Le righe della tabella corrispondono alle istanze del problema. Le colonne mostrano, per ogni algoritmo, il peso della soluzione trovata, il tempo di esecuzione e l'errore relativo calcolato come $(SoluzioneTrovata - SoluzioneOttima)/SoluzioneOttima$. Potete aggiungere altre informazioni alla tabella che ritenete interessanti. 
\image{1}{estab}{Esempio di tabella riportante i risultati ottenuti}
\subsubsection{Svolgimento}
\label{tab}
\begin{center}
	\scriptsize
	\begin{longtable}{|c|c|c|c|c|c|c|c|c|c|}	
	\hline
		\multirow{2}{*}{\textbf{Istanza}} & \multicolumn{3}{c|}{\textbf{Held-Karp}} & \multicolumn{3}{c|}{\textbf{Cheapest Insertion}} & \multicolumn{3}{c|}{\textbf{Triangle\_TSP}} \\ \cline{2-10}
		 &\textbf{Soluzione}& \textbf{Tempo (s)} & \textbf{Errore} & \textbf{Soluzione}& \textbf{Tempo (s)} & \textbf{Errore} & \textbf{Soluzione}& \textbf{Tempo (s)} & \textbf{Errore} \\ \hline
		\endfirsthead
		\multicolumn{10}{|c|}%
		{\tablename\ \thetable\ \ --\  \textit{continuazione della pagina precedente}} \\
		\hline
		\multirow{2}{*}{\textbf{Istanza}} & \multicolumn{3}{c|}{\textbf{Held-Karp}} & \multicolumn{3}{c|}{\textbf{Held-Karp}} & \multicolumn{3}{c|}{\textbf{Held-Karp}} \\ \cline{2-10}
		 &\textbf{Soluzione}& \textbf{Tempo (s)} & \textbf{Errore} & \textbf{Soluzione}& \textbf{Tempo (s)} & \textbf{Errore} & \textbf{Soluzione}& \textbf{Tempo (s)} & \textbf{Errore} \\ \hline
		\endhead
		\hline \multicolumn{10}{|r|}{\textit{Continua nella pagina seguente}} \\
		\endfoot
		\endlastfoot
		berlin52.tsp & 17441 & 120,0174 & 131,25\% & 9004 & 0,0096 & 19,38\%& 10402 &0,0074 &37.92\% \\ \hline
		burma14.tsp & 3323 & 0,2029 & 0,0\% & 3588 & 0,0003 & 7,97\% &4003 & 0,0002&20.46\% \\ \hline
		ch150.tsp & 47935 & 120,95 & 634,30\% & 7998 & 0,0678 & 22,52\% &9126 &0,02343 &39.80\%  \\ \hline
		d493.tsp & 111947 & 120,4769&219,83\% & 39969 & 0,4976 & 14,19\% &45300 &0,1063 &29.42\% \\ \hline
		dsj1000.tsp & \scriptsize 551274242 &120,0012 &2854,36\% & 22291165 & 7,5382 & 19,46\% &25526005 &0,5358 &36.80\% \\ \hline
		eil51.tsp&986 &120,0005 &131,46\% & 494 &0,0006 & 15,96\% &614 & 0,0009 &44.13\% \\ \hline
		gr202.tsp&55127 &119,9998 &37,26\% & 46480 & 0,0406 & 15,74\% &52615 & 0,0189& 31.01\%\\ \hline
		gr229.tsp&176212 & 120,0008&30,91\% & 153896 & 0,0493 & 14,33\% &179335 &0,0249 &33.23\% \\ \hline
		kroA100.tsp& 164223 & 120,0003 & 671,65\% & 24942 & 0,0038 & 17,20\%&30536 &0,0014 &43.48\% \\ \hline
		kroD100.tsp& 144125 &120,0012 &576,83\% &25204 & 0,0036 & 18,36\% &28599 &0,0017 &34.31\% \\ \hline
		pcb442.tsp& 202233&120,4061 &298,27\% & 60834 & 0,3954 & 19,80\% &68841 &0,0901 & 35.57\%\\ \hline
		ulysses16.tsp&6859 &0,5584 & 0,0\% & 7368 & 0,0001 & 7,42\%&7788 & 0,0001 & 13.54\%\\ \hline
		ulysses22.tsp&7013 &59,4115 &0,0\% &  7709 & 0,0001 & 9,92\% &8308 &0,0001 &18.47\% \\ \hline		 
     \caption{Risultati dei tre algoritmi implementati rispetto alla domanda 1}
	\end{longtable}
\end{center}

\subsubsection{Held-Karp}
Mobilitati da curiosità abbiamo provato a variare il tempo di calcolo a disposione dell'algoritmo di Held e Karp, per vedere se le soluzioni che impiegavano più di due minuti ritornavano una soluzione migliore, oppure se, con meno tempo a disposizione, la soluzione ritornata risultava peggiore. Le tabelle sottostanti riportano quello che è stato ottenuto.
\paragraph{Held-Karp con 1 minuto}
\begin{center}
	\begin{longtable}{|c|c|c|c|}	
	\hline
		\multirow{2}{*}{\textbf{Istanza}} & \multicolumn{3}{c|}{\textbf{Held-Karp}} \\ \cline{2-4}
		 &\textbf{Soluzione}& \textbf{Tempo (s)} & \textbf{Errore} \\ \hline
		\endfirsthead
		\multicolumn{4}{|c|}%
		{\tablename\ \thetable\ \ --\  \textit{continuazione della pagina precedente}} \\
		\hline
		\multirow{2}{*}{\textbf{Istanza}} & \multicolumn{3}{c|}{\textbf{Held-Karp}} \\ \cline{2-4}
		 &\textbf{Soluzione}& \textbf{Tempo (s)} & \textbf{Errore} \\ \hline
		\endhead
		\hline \multicolumn{4}{|r|}{\textit{Continua nella pagina seguente}} \\
		\endfoot
		\endlastfoot
berlin52 & 17441 & 60,706 & 131,25\% \\ \hline
burma14 & 3323 & 0,1014 & 0,00\% \\ \hline
ch150 & 47935 & 59,999 & 634,29\% \\ \hline
d493 & 111947 & 60,598 & 219,83\% \\ \hline
dsj1000 & 551274242 & 60,001 & 2854,35\% \\ \hline
eil51 & 986 & 60,000 & 131,45\% \\ \hline
gr202 & 55127 & 60,000 & 37,26\% \\ \hline
gr229 & 176212 & 59,999 & 30,91\% \\ \hline
kroA100 & 165018 & 60,000 & 675,38\% \\ \hline
kroD100 & 146158 & 59,999 & 586,38\% \\ \hline
pcb442 & 202517 & 59,999 & 298,82\% \\ \hline
ulysses16 & 6859 & 0,3550 & 0,00\% \\ \hline
ulysses22 & 7013 & 53,181 & 0,00\% \\ \hline
		\caption{Risultati dell'algoritmo Held-Karp concedendo 1 minuto}
	\end{longtable}
\end{center}\vspace{-40pt}
\paragraph{Held-Karp con 5 minuti}
\begin{center}
	\begin{longtable}{|c|c|c|c|}	
	\hline
		\multirow{2}{*}{\textbf{Istanza}} & \multicolumn{3}{c|}{\textbf{Held-Karp}} \\ \cline{2-4}
		 &\textbf{Soluzione}& \textbf{Tempo (s)} & \textbf{Errore} \\ \hline
		\endfirsthead
		\multicolumn{4}{|c|}%
		{\tablename\ \thetable\ \ --\  \textit{continuazione della pagina precedente}} \\
		\hline
		\multirow{2}{*}{\textbf{Istanza}} & \multicolumn{3}{c|}{\textbf{Held-Karp}} \\ \cline{2-4}
		 &\textbf{Soluzione}& \textbf{Tempo (s)} & \textbf{Errore} \\ \hline
		\endhead
		\hline \multicolumn{4}{|r|}{\textit{Continua nella pagina seguente}} \\
		\endfoot
		\endlastfoot
berlin52 & 17441 & 303,951 & 131,25\% \\ \hline
burma14 & 3323 & 0,15270 & 0,00\% \\ \hline
ch150 & 47935 & 316,682 & 634,29\% \\ \hline
d493 & 111947 & 308,136 & 219,83\% \\ \hline
dsj1000 & 551274242 & 306,492 & 2854,35\% \\ \hline
eil51 & 984 & 300,463 & 130,98\% \\ \hline
gr202 & 55127 & 305,269 & 37,26\% \\ \hline
gr229 & 176212 & 312,290 & 30,91\% \\ \hline
kroA100 & 162606 & 308,605 & 664,05\% \\ \hline
kroD100 & 144125 & 314,625 & 576,83\% \\ \hline
pcb442 & 202044 & 300,972 & 297,89\% \\ \hline
ulysses16 & 6859 & 0,40430 & 0,00\% \\ \hline
ulysses22 & 7013 & 57,7751 & 0,00\% \\ \hline
		\caption{Risultati dell'algoritmo Held-Karp concedendo 5 minuti}
	\end{longtable}
\end{center}\vspace{-40pt}
\paragraph{Held-Karp con 10 minuti}
\begin{center}
	\begin{longtable}{|c|c|c|c|}	
	\hline
		\multirow{2}{*}{\textbf{Istanza}} & \multicolumn{3}{c|}{\textbf{Held-Karp}} \\ \cline{2-4}
		 &\textbf{Soluzione}& \textbf{Tempo (s)} & \textbf{Errore} \\ \hline
		\endfirsthead
		\multicolumn{4}{|c|}%
		{\tablename\ \thetable\ \ --\  \textit{continuazione della pagina precedente}} \\
		\hline
		\multirow{2}{*}{\textbf{Istanza}} & \multicolumn{3}{c|}{\textbf{Held-Karp}} \\ \cline{2-4}
		 &\textbf{Soluzione}& \textbf{Tempo (s)} & \textbf{Errore} \\ \hline
		\endhead
		\hline \multicolumn{4}{|r|}{\textit{Continua nella pagina seguente}} \\
		\endfoot
		\endlastfoot
berlin52 & 17441 & 603,349 & 131,25\% \\ \hline   
burma14 & 3323 & 0,15163 & 0,00\% \\ \hline
ch150 & 47885 & 606,479 & 633,53\% \\ \hline
d493 & 111947 & 603,658 & 219,83\% \\ \hline
dsj1000 & 550719964 & 611,369 & 2851,38\% \\ \hline
eil51 & 963 & 601,293 & 126,05\% \\ \hline
gr202 & 55125 & 605,391 & 37,26\% \\ \hline
gr229 & 175982 & 603,008 & 30,74\% \\ \hline
kroA100 & 161443 & 699,899 & 658,58\% \\ \hline
kroD100 & 143960 & 605,555 & 576,05\% \\ \hline
pcb442 & 202044 & 603,171 & 297,89\% \\ \hline
ulysses16 & 6859 & 0,40640 & 0,00\% \\ \hline
ulysses22 & 7013 & 58,3268 & 0,00\% \\ \hline
		\caption{Risultati dell'algoritmo Held-Karp concedendo 10 minuti}
	\end{longtable}
\end{center}

\paragraph{Conclusioni}
La variazione del tempo concesso per il calcolo dell'algoritmo \texttt{HeldKarp} non ha avuto alcun miglioramento né peggioramento per sei istanze di TSP su tredici. Per tre di loro, precisamente \texttt{burma14.tsp, ulysses16.tsp e ulysses22.tsp}, questo accade perché la soluzione ottima viene già trovata entro un intervallo di tempo relativamente basso, quindi non considerato perché ritenuto poco interessante: la più \quotes{impegnativa} delle tre, \texttt{ulysses22.tsp}, è in grado di ritornare la soluzione ottima in poco meno di un minuto. Le altre due in meno di un secondo.\eqcapo
Per quanto riguarda le rimanenti sette istanze di TSP, vale a dire \texttt{ch150.tsp, dsj1000.tsp, eil51.tsp, gr202.tsp, gr229.tsp, kroA100.tsp, kroD100.tsp} invece si è registrato un aumento della qualità della soluzione ritornata, all'aumentare del tempo di computazione concesso. I grafici sottostanti riportano i dati inseriti nelle tabelle di cui sopra, mostrando più chiaramente la differenza degli errori relativi calcolati come $(SoluzioneTrovata - SoluzioneOttima)/SoluzioneOttima$,  ripartito in base alle diverse tempistiche assegnate.
\paragraph*{ch150.tsp}
\image{0.7}{ch150}{Variazione dell'errore relativo in ch150}
\paragraph*{dsj1000.tsp}
\image{0.7}{dsj1000}{Variazione dell'errore relativo in dsj1000}
\paragraph*{eil51.tsp}
\image{0.7}{eil51}{Variazione dell'errore relativo in eil51}
\paragraph*{gr202.tsp}
\image{0.7}{gr202}{Variazione dell'errore relativo in gr202}
\textbf{Nota:} in questo grafico, essendo troppo piccola la variazione, i due valori riportati sull'asse delle ordinate vanno sommati con il numero scritto in notazione scientifica in alto a sinistra.
\paragraph*{gr229.tsp}
\image{0.7}{gr229}{Variazione dell'errore relativo in gr229}
\paragraph*{kroA100.tsp}
\image{0.7}{kroA100}{Variazione dell'errore relativo in kroA100}
\paragraph*{kroD100.tsp}
\image{0.7}{kroD100}{Variazione dell'errore relativo in kroD100}
\mbox{}\eqcapo

Come ci si poteva aspettare, concedendo più tempo di computazione all'algoritmo \texttt{HeldKarp}, abbiamo registrato una probabilità non banale che l'algoritmo migliori le sue \emph{performance}: 
\begin{itemize}
	\item Non considerando le tre istanze di TSP in cui l'algoritmo riesce a trovare la soluzione ottima sotto al minuto, si rimane quindi con dieci istanze: \texttt{berlin52.tsp, ch150.tsp, d493.tsp, dsj1000.tsp, eil51.tsp, gr202.tsp, gr229.tsp, kroA100.tsp, kroD100.tsp} e \texttt{pcb442.tsp}.
	\item Sette di loro hanno un miglioramento della soluzione ritornata $\Rightarrow$ \textbf{70\%} delle possibilità che \texttt{HeldKarp} restituisca una soluzione più vicina alla soluzione ottima all'aumentare del tempo concesso.
\end{itemize}
Seppure ci sia chiaro che non sia corretto derivare giudizi generali di qualsiasi tipo avendo a disposizione così poche istanze di TSP, riteniamo di aver ottenuto una percentuale davvero alta, di cui si debba tenere conto, risultato sperimentale che conferma ciò che ci aspettavamo.\acapo

Crediamo che, alla luce di quanto detto finora, sia il caso di chiederci: vale la pena concedere fino a dieci minuti per trovare una soluzione migliore?
\begin{itemize}
	\item La risposta che ci siamo dati è, non sorprendentemente, \textbf{dipende}: se il grafo ricavato dall'istanza di TSP possiede un numero di nodi $n$:
	\begin{itemize}
		\item $0< n \leq 22 $ allora assegnare più tempo a \texttt{HeldKarp} \textbf{è indifferente}, in quanto l'algoritmo ritorna una soluzione ottima in meno di un minuto (quindi si ferma in ogni caso);\newpage
		\item $23 \leq n < 52$ allora assegnare più tempo a \texttt{HeldKarp}\textbf{ probabilmente non conviene}, perché impiegando invece l'euristica \texttt{CheapestInsertion} si ottengono risultati statisticamente migliori (vedi tabella \ref{tab}). Ad ogni modo, occorre precisare che per valori di $n$ immediatamente successivi a 22, invece, \textbf{probabilmente conviene} lasciare più tempo all'algoritmo, perché potrebbe ritornare una soluzione ottima entro 10 minuti.
		\item $ n\geq 52$ allora assegnare più tempo a \texttt{HeldKarp} \textbf{sicuramente non conviene}, perché abbiamo sperimentato che, per ogni istanza di TSP, risultati migliori si possono ottenere impiegando l'euristica \texttt{CheapestInsertion}.
	\end{itemize}
\end{itemize}