\section{Risultati degli algoritmi}
Questa sezione risponderà alla Domanda 1: verranno riportati sotto forma di tabella i risultati dei costi per il problema TSP dei grafi richiesti, il tempo di esecuzione di ogni algoritmo e l'errore relativo rispetto alla soluzione esatta.

\subsection{Specifiche Hardware dei calcolatori utilizzati}
Dato il considerevole divario di prestazioni ottenute dalle due diverse macchine, gli studenti hanno ritenuto opportuno riportare le differenti tempistiche impiegate per il calcolo dei costi degli MST.
\begin{center}
	\begin{longtable}{ r | c | c } %\hline
	\multicolumn{1}{c|}{\textbf{Caratteristica}} &\textbf{PC di Nicola}&\textbf{PC di Federico}\\ \hline 
	\endfirsthead
	\rowcolor{white}
	\multicolumn{3}{|r|}{\textit{-- continuazione da pagina precedente}} \\ \hline 
	\endhead
	\hline
	\rowcolor{white} 
	\multicolumn{3}{|r|}{{\textit{-- continua a pagina successiva}}} \\
	\endfoot
	\endlastfoot
	%table input should begin here
	Architettura & 64 bit & 64 bit \\
	Nome processore & Intel i5-7300HQ & Intel i7-8750H \\
	Numero core & 4 & 6\\
	Numero thread & 4 & 12 \\
	Range velocità di clock [GHz] & 2.50 - 3.50 & 2.20 - 4.10\\
	Dimensione cache L1 [KiB] & 256 & 384\\
	Dimensione cache L2 [MiB] & 1 & 1.5\\
	Dimensione cache L3 [MiB] & 6 & 9\\
	Dimensione RAM [GiB] & 8 & 31.2\\  \hline
	\caption{Specifiche dei calcolatori utilizzati\\ (Fonte: \url{http://intel.com}).}
	\end{longtable}
\end{center} 

\subsection{HeldKarp}
L'algoritmo non presenta variazioni nell'implementazione rispetto all'algoritmo mostrato a lezione, dunque possiede una complessità di \comp{n^2 2^n}.

%TODO



