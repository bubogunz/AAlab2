\subsection{Tree\_TSP}

L'algoritmo crea il circuito per TSP attraverso la lista preordinata dei nodi del MST ottenuto dall'algoritmo \texttt{Kruskal}. Tale algoritmo costruisce un albero la cui radice è un'istanza della classe \texttt{Node}, che non ha niente a che fare con la classe \texttt{Graph} ed è utilizzata esclusivamente dall'algoritmo \texttt{Kruskal}. Un discorso analogo vale per la classe \texttt{Edge} usata solamente dall'algoritmo \texttt{Kruskal} per ottenere una lista ordinata per peso dei lati del grafo.

Dato $n=|V|$ e $m=|E|$, l'algoritmo \texttt{Tree\_TSP}, dato utilizza \texttt{Kruskal}, ha una complessità \comp{m\log n}.

L'algoritmo è 2-approssimazione per TSP, e per i grafi forniti ottiene risultati con un errore relativo al massimo del 45\%.   

\begin{table}[H]
	\centering
	\begin{tabular}{|c|c|c|c|c|}
		\hline
		\textbf{N.} & \textbf{Name Graph} & \textbf{TSP cost} & \textbf{Time (s)} & \textbf{Error (\%)}\\ 
		\hline
		1 & berlin52 & 10402 & 0.0083738 & 37.92\\
		\hline
		2 & burma14 & 4003 & 2.059E-4 & 20.46\\
		\hline
		3 & ch150 & 9126 & 0.0253737 & 39.80\\
		\hline
		4 & d493 & 45300 & 0.1079314 & 29.42\\
		\hline
		5 & dsj1000 & 25526005 & 0.4340513 & 36.80\\
		\hline
		6 & eil51 & 614 & 5.553E-4 & 44.13\\
		\hline
		7 & gr202 & 52615 & 0.0187077 & 31.01\\
		\hline
		8 & gr229 & 179335 & 0.0203114 & 33.23\\
		\hline
		9 & kroA100 & 30536 & 0.0014703 & 43.48\\
		\hline
		10 & kroD100 & 28599 & 0.001964 & 34.31\\
		\hline
		11 & pcb442 & 68841 & 0.0833932 & 35.57\\
		\hline
		12 & ulysses16 & 7788 & 6.06E-5 & 13.54\\
		\hline
		13 & ulysses22 & 8308 & 9.25E-5 & 18.47\\
		\hline
	\end{tabular}
	\caption{Risultati dell'algoritmo \texttt{Tree\_TSP}}
\end{table}

\image{1}{Tree_TSP}{Performance dell'algoritmo \texttt{Tree\_TSP}}